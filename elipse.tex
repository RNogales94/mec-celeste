\documentclass[12pt]{article}
\usepackage[utf8x]{inputenc}
\usepackage{graphicx}
\usepackage{multirow}
\usepackage{hhline}
\usepackage{booktabs}
\usepackage{vmargin} %cambia el margen
\usepackage{amsmath,amsthm}
\usepackage{amsfonts}
\usepackage{float}
% or
%\usepackage{amssymb}
\pretolerance=2000
\tolerance=3000

%Paquetes para grafos
\usepackage{tikz}
\usepackage{pgfplots}
\usepackage{color}

%Definición de letras caligráficas
\renewcommand{\H}{\mathcal{H}}
\newcommand{\X}{\mathcal{X}}
\newcommand{\Y}{\mathcal{Y}}
\newcommand{\D}{\mathcal{D}}
\newcommand{\A}{\mathcal{A}}
\renewcommand{\P}{\mathcal{P}}



\title{
    Mecánica Celeste\\
    \large La Elipse
}


\author{
        Nuria Angulo Fuentes \\
        Cristina Callejón Maleno \\
        Verónica Guerrero Contreras \\
        Ainoa Muros Quesada \\
        Rafael Nogales Vaquero \\       
        María del Mar Ruiz Martín \\        
        Laura Prados Sáez \\
        Grado de Matemáticas\\
        Universidad de Granada - UGR\\
        \underline{Spain}\\
        Curso 2017/2018
        \date{}
}



\begin{document}
\maketitle
\begin{center}
\includegraphics[scale=0.45]{images/escudo.jpg}
\end{center}

\pagebreak

\section{Introducción}



\section{La elipse}


\textbf{Definición:} se define la elipse como el lugar geométrico de los puntos de $\mathbb{R}^2$ cuya suma de las distancias a dos puntos dados, llamados focos, es constante. Es decir:\\

\begin{equation} E = \{x \in \mathbb{R}^2 : \|x-F_1\| + \|x-F_2\| = d\}  \label{eq1}\end{equation}\\

donde $F_1, F_2$ son los focos y d es una constante positiva. \\

Buscamos expresar la elipse como 

\begin{equation} \{ x \in \mathbb{R}^2 : |x| +  <x,e> = K \}  \label{eq2}\end{equation}\\

para determinados $e\in \mathbb{R}^2$, $K>0$ constantes. Para ello nos centraremos en aquellas elipses que tienen el foco $F_1$ en el punto (0,0). El resto de elipses, al ser traslaciones de las anteriores, las podremos expresar como 

$$  \{ x + (F_{1,1},F_{1,2}): x \in \mathbb{R}^2, |x| +  <x,e> = K \} $$  \\
    
donde $(F_{1,1},F_{1,2})$ es la posición del foco $F_1$.\\



Sea $x \in E$, entonces podemos afirmar que se verifica:

$$\|x\| + \|F_2 - x\| = d \Leftrightarrow   \|F_2 - x\| = d - \|x\| \Rightarrow (\|F_2 - x\|)^2 = (d - \|x\|)^2 \Leftrightarrow $$
$$\Leftrightarrow \|x\|^2 + \|F_2\|^2 - 2 <F_2, x> = d^2 + \|x\|^2 -2 <d,\|x\|>  \Leftrightarrow $$
$$\Leftrightarrow \|F_2\|^2 - 2 <F_2, x> = d^2 -2 <d,\|x\|> \Leftrightarrow$$
$$\Leftrightarrow \frac{\|F_2\|^2}{2d} - \frac{<F_2, x>}{d} = \frac{d}{2} -\|x\| \Leftrightarrow \|x\| +  < -\frac{F_2}{d}, x> = \frac{d^2-\|F_2\|^2}{2d}$$\\


Definiendo 

\begin{equation}e := -\frac{1}{d}F_2, K:=\frac{d^2-\|F_2\|^2}{2d}  \label{eq3}\end{equation}\\

obtenemos la expresión buscada.\\


A continuación vamos a ver que todo conjunto de la forma 

$$ \{ x \in \mathbb{R}^2 : |x| +  <x,e> = K \}, e \in \mathbb{R}^2, \|e\| < 1, K > 0$$  \\


representa una elipse con el foco $F_1 = (0,0)$ para determinada distancia $d > 0$ y foco $F_2\in \mathbb{R}^2$.\\


En virtud a las ecucaciones dadas en $(\ref{eq3})$, buscamos $F_2$, d de la forma 
\begin{equation}F_2 = -ed  \label{eq4}\end{equation}
\begin{equation}2dK -d^2 + \|F_2\|^2 = 0  \label{eq5}\end{equation}\\

Sustituyendo $(\ref{eq4})$ en $(\ref{eq5})$ obtenemos:



$$2dK -d^2 + \|ed\|^2 = 0 \Leftrightarrow d(2K - d + d\|e\|^2) = 0 \Leftrightarrow
 \left\{
 \begin{array}{ll}
                  d = 0\\
                  o\\
                  2K - d + d\|e\|^2 = 0
                \end{array}
              \right.$$


Desechamos la primera opción, puesto que sustituyendo en $(\ref{eq4})$ obtendríamos que $F_2$ es el punto (0,0), coincidiendo con $F_1$, y por tanto, no tendríamos una elipse. Desarrollamos entonces la segunda posibilidad:

\begin{equation}2K - d + d\|e\|^2 = 0 \Leftrightarrow d = \frac{2K}{1 - \|e\|^2}  \label{eq6}\end{equation}


Y sustituyendo finalmente en $(\ref{eq4})$ obtenemos la siguiente expresión: 


\begin{equation} F_2 =  \frac{2K}{\|e\|^2 - 1} e  \label{eq7}\end{equation}


Despejando e y K de $(\ref{eq4})$ y $(\ref{eq5})$ respectivamente, obtenemos las siguientes expresiones:

\begin{equation}e = -\frac{1}{d}F_2, K=\frac{d^2-\|F_2\|^2}{2d}  \label{eq8}\end{equation}\\

Pasamos ahora a sustituir en la expresión estos valores:

$$\|x\| +  <x,e> = K \Leftrightarrow  \|x\| +  <x,-\frac{1}{d}F_2> = \frac{d^2-\|F_2\|^2}{2d}  \Leftrightarrow $$
$$\Leftrightarrow 2d\|x\| +  <x,-2F_2> = d^2-\|F_2\|^2 \Leftrightarrow \|F_2\|^2  +  -2<x,F_2> = d^2 + 2d\|x\|\Leftrightarrow $$
$$\Leftrightarrow \|x\|^2 + \|F_2\|^2  +  -2<x,F_2> = \|x\|^2 + d^2 + 2d\|x\|\Leftrightarrow (\|x - F_2\|)^2   = (\|x\|- d)^2\Leftrightarrow$$
$$ \Leftrightarrow\left\{
 \begin{array}{ll}
                  \|x - F_2\|   = \|x\|- d\\
                  o\\
                  \|x - F_2\|   = d - \|x\|
                \end{array}
              \right.$$\\


Veamos por reducción al absurdo que no es posible que se de la primera igualdad:\\

$$
\|x\| - d < \|x\| - \|F_2\| \leq \|x - F_2\| = \|x\| - d
$$\\


Por tanto, podemos asegurar que se da la igualdad $\|x - F_2\|   = d - \|x\|$. Y entonces

$$ \|x\| + \|x - F_2\|   = d $$

Concluímos así que la expresión inicial define una elipse con foco $F_1 = (0,0)$ y como parámetros d y $F_2$ los dados en $(\ref{eq6})$ y $(\ref{eq7})$.


\end{document}

